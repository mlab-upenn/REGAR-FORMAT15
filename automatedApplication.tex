\subsection{Automatic rule application}
\label{automatedApplication}

Because we start from a set of initial models, and have a set of abstraction rules that can be applied to any given model, we have a choice of which rules to apply to which models, and the order in which to apply them.
E.g. consider the labeled graph of Fig.???, to which both rules R6 and R7 are applicable.
Depending on which rule is applied first, we end up with different abstract models.
In this section we propose a measure of abstractness, and sketch how an optimal order of application can be found as the solution of a Mixed-Integer Nonlinear Program (MINLP).
Details can be found in the technical report. 

First, we observe that when a rule is applied to a labeled graph $G=(V,E,A)$, it either decreases the number of vertices, or the number of edges, or it enlarges the invariant and/or guard sets of some of the automata (by manipulating the parameters as done, e.g., by Rule 4).
We define a measure of the abstraction power $\alpha(R)$ of a rule $R$ to be
\[\alpha(R) = |V(G)| - |V(R(G))| + |E(G)| - |E(R(G))| + \sum_{x \in \Ec(G)\cap \Ec(R(G))}|\theta^x - R(\theta^x)|\] 
Intuitively, a rule $R$ is more abstract than rule $R'$ if it removes more elements of enlarges the parameter ranges more.

Rule application must respect the following constraints, all of which can be encoded as the constraints of a MINLP:
\begin{enumerate}
	\item Only one rule can be applied at a time.
	\item A rule can only be applied to a (sub)graph with a given structure, and whose parameters respect certain conditions.
	\item When a rule is applied to a subgraph, it may disable future rule applications to this subgraph, either because a) it changed the subgraph's structure, or b) it updated the subgraphs's parameters.
\end{enumerate}
We define binary variables $a_{isj}$ such that $a_{isj} = 1$ iff at the $j^{th}$ time step, we apply rule $R_i$ to subgraph $K_s \subset G$.
The first constraint, for example, can be encoded as
\[\forall j=1,\ldots, N, \sum_{i,s}a_{isj} \leq 1\]

Proceeding in this manner, we formulate a MINLP whose solution gives an optimal order of rule application, in the sense of maximizing the abstractness of the final model.
Then model-checking can be performed on this most abstract model, which, by construction, covers the most behavior.