\section{Closed-loop Model Checking Using the Abstraction Tree}
After the abstraction tree is built, it can be used for closed-loop model checking. The next question is how to navigate through the abstraction tree so that the most appropriate model(s) are selected for different requirements, and provide the most concrete counter-examples, possibly under multiple physiological conditions, for the physicians to determine the validity of the counter-examples. %\cite{uppaal}
\subsection{Select Initial Abstraction(s) Appropriate For the Requirement}
%Physiological requirements are in general conditional in the sense that they require conditions to hold under certain open-loop physiological constraints. These constraints map to parameters for certain transitions of the physiological models, which may be merged during the abstraction process.
For a requirement $Req$ and its auxiliary monitor $Mon$, an environment model $M_E$ is appropriate for $Req$ if the variables contained in $M_E$ is a superset of the variables in the requirement and its auxiliary monitor: 
$$Var(Req)\subseteq Var(M_E)\cup Var(Mon)$$
Algorithm 1 (\figref{algorithm}) selects the most abstract models from the abstraction tree $HM\_tree$ that are appropriate for a requirement $Req$.
\begin{figure}[!t]
		\centering
		\includegraphics[width=0.9\textwidth]{figs/algorithm.pdf}
		%\vspace{-5pt}
		\caption{\small Algorithms for closed-loop model checking with abstraction tree}
		  %\vspace{-15pt}
		\label{fig:algorithm}
\end{figure}
%The most abstract model in the abstraction tree is built to cover the input space to the device as much as possible, thus it may not have enough details to constrain the behaviors of the model according to the constraints in the requirement. Thus the first step of closed-loop model checking is to select the most abstract models which are appropriate for the requirement. A model $M$ is appropriate for a requirement $Req$ if the environment transitions mentioned in the requirement, denoted as $EnvT(Req)$, is a subset of the environment transitions associated with the model $EnvT(M$). The following algorithm finds the most abstract heart models in the abstraction tree $HM\_tree$ that are appropriate for a requirement $Req$.
%\begin{Verbatim}
%Algorithm 1
%function [HM]=eligible(HM_tree,Req)
%BM = root of HM_tree
%while (BM is not empty)
 %For every model M in BM
  %If (Var(Req) is a subset of Var(M)+Var(Mon))
   %Remove M from BM
   %save M in HM
  %else
   %add children of M in BM
  %endif
 %endfor
%endwhile
%Return HM
%\end{Verbatim}
%BM = all models in HM_tree with all Prop(Req)
%For all M in BM
 %while (in the parent of M no behavior in Prop(Req) is abstracted with other behaviors)
	 %M = parent of M
 %endwhile
 %save M in HM
 %endfor
\subsection{Obtaining Concrete Counter-examples with Unambiguous Physiological Context}
%One challenge for modeling physiological environment of medical devices is the large variety of physiological conditions that the devices may encounter. %It is impossible to exhaustively model the conditions with concrete physiological models. 
%The set of concrete models (the leave nodes in the abstraction tree) only represent a subset of all possible physiological conditions, thus model checking the device on all of the concrete models is not enough to guarantee the satisfaction of the requirement. By applying physiological abstraction rules, additional physiological-relevant behaviors are incorporated into the abstract models by over-approximation, providing more coverage. 
%However, if model checking on an abstract model returns a counter-example, if is difficult to determine whether the counter-example is a valid execution. Due to the incomplete nature of the set of concrete models, if the counter-example cannot be concretized on all the concrete models, it does not mean it is invalid. Therefore it is up to the physician to decide whether a counter-example is valid or not. The algorithm below explore the abstraction tree during model checking and provide the physician the most concrete counter-example(s), if there exists any. 
Upon requirement violation, an abstract counter-example is returned during closed-loop model checking. The counter-example may contain context ambiguities that should be resolved before sending to the physician. An abstraction tree contains the information of how different environment conditions are over-approximated by more abstract models. By exploring the abstraction tree, the most concrete yet unambiguous counter-examples can be sent to the physicians for analysis. Algorithm 2 (\figref{algorithm}) describes the process for closed-loop model checking with an abstraction tree.
%\begin{Verbatim}
%Algorithm 2
%Input: system model PM, abstraction tree for environment HM_tree, requirement Req
%Output: Counter examples CE and corresponding model refinements, if any
%[HM]=eligible(HM_tree,Req);
%Mc= HM;
 %while (Mc is not empty)
  %For all M in Mc
   %[satisfied,CE]=ModelChecking(M,PM,Req);
	 %Remove M from Mc
	 %If satisfied==0
	  %add the children of M to Mc
		%cache CE
	%else
		%save CE from the parent model
	%endif
 %endfor
%endwhile
%Return all saved CEs and their corresponding models
%\end{Verbatim}

\begin{figure}[!t]
		\centering
		\includegraphics[width=0.8\textwidth]{figs/PM_timers.pdf}
		%\vspace{-5pt}
		\caption{\small Basic timers for a dual chamber pacemaker. AS: Atrial Sense, VS: Ventricular Sense, AP: Atrial Pacing, VP: Ventricular Pacing.}
		  %\vspace{-15pt}
		\label{fig:PM_timers}
\end{figure}
\section{Case Study: Closed-loop Model Checking of a Dual Chamber Pacemaker}
\subsection{Pacemaker Model}
A pacemaker diagnoses heart conditions and deliver electrical pacing to the heart according to the timing intervals among timing events from the heart and the pacemaker itself. In this section we use a simple dual chamber pacemaker as example for closed-loop model checking. The detailed UPPAAL timed automata implementation of the model can be found in \cite{sttt13}. A dual chamber pacemaker has several basic timers, which are shown in \figref{PM_timers}:\\
\textbf{Atrial Escape Interval (AEI)} defines the maximum interval between the last ventricular event (VS,VP) to an atrial event (AS,AP). If no AS happened before AEI timer runs out, atrial pacing (AP) is delivered to the heart (Marker 1 in \figref{PM_timers}). \\
\textbf{Atrio-Ventricular Interval (AVI)} defines the maximum interval between the last atrial event (AS,AP) to an ventricular event (VS,VP). If no VS happened before AVI timer runs out, and the time since the last ventricular event (VS,VP) is no less than URI, ventricular pacing (VP) is delivered to the heart (Marker 3 in \figref{PM_timers}).
\textbf{Post-Ventricular Atrial Refractory Period (PVARP) and Ventricular Refractory Period (VRP)} define the minimum period that a AS or VS can happen since the last ventricular event (VS,VP). 

\subsection{Requirement Encoding}
The requirement below is designed to prevent the pacemaker from pacing too fast.\\
\textsf{If intervals between self-activations of the atria is between 300ms to 1000ms, the intervals between ventricular paces should be no shorter than 500ms.}

With parameter mapping and a monitor $M_{sing}(VP,500,\infty)$, the requirement can be translated to:
$$Req1: N_A.loc=Rest \&\& N_A.t\in [300,1000] \Rightarrow \neg M_{sing}.Err$$

 \subsection{Choosing Appropriate Heart Model For the Requirement}
To verify the closed-loop system with pacemaker model $PM$ and heart model abstraction tree $HM\_tree$ (\figref{HM_abs}) against requirement $Req1$, we start by searching for the most abstract appropriate models from the abstraction tree. We call the function specified in Algorithm 1: $[HM]=eligible(HM\_tree,Req1)$. The variables in the requirement and the corresponding monitor are:
$$Var(Req1)\cup Var(M_{sing})=\{N_A.t,N_A.loc, VP\}$$

At the root level heart model $H_{all}$, we have $\{N_A.t,N_A.loc\}\not\in Var(H_{all})$. As the result, $H_{all}$ is not appropriate for $Req1$. For all the children of $H_{all}$: $H_n'',H_{at}'''',H_{vt}'''$, we have $Var(Req1)\cup Var(M_{sing})\subseteq Var(H_n'')=Var(H_{at}'''')=Var(H_{vt}''')$, thus these 3 heart models are outputted as the most abstract models that are appropriate for $Req1$.
\begin{figure}[!t]
	\centering
	\includegraphics[width=0.5\textwidth]{figs/abs_sim.pdf}
	%\vspace{-5pt}
	\caption{\small Abstraction Rule Application Example}
	%\vspace{-15pt}
	\label{fig:abs_exam}
\end{figure}
\subsection{Providing Unambiguous Counter-examples to the Physicians}
After we choose the appropriate models for $Req1$, we have: 
$$HM=\{H_n'',H_{at}'''',H_{vt}'''\}$$
Then we run Algorithm 2. By model checking on all 3 initial models in UPPAAL we have: 
$$[1,[]]=ModelChecking(H_n'',PM,Req1)$$
 $$[0,CE_1]=ModelChecking(H_{at}'''',PM,Req1)$$
$$[0,CE_2]=ModelChecking(H_{vt}''',PM,Req1)$$
For the two heart models $H_{at}'''',H_{vt}'''$ in which the requirement is violated, the algorithm keeps going down the abstraction tree, and upon termination counter-examples are returned for the following heart models:
$$H_{at};H_{pvc};H_{af};H_{avn};H_{afib}$$
%$$[0,CE_{at}]=ModelChecking(H_{at},PM,Req1)$$
%$$[0,CE_{pvc}]=ModelChecking(H_{pvc},PM,Req1)$$
%$$[0,CE_{af}]=ModelChecking(H_{af},PM,Req1)$$
%$$[0,CE_{avn}]=ModelChecking(H_{avn},PM,Req1)$$
%$$[0,CE_{afib}]=ModelChecking(H_{afib},PM,Req1)$$

\begin{figure}[!t]
		\centering
		\includegraphics[width=0.9\textwidth]{figs/case.pdf}
		%\vspace{-5pt}
		\caption{\small Counter-examples}
		  %\vspace{-15pt}
		\label{fig:CE}
\end{figure}

The counter-examples are then sent to physicians for analysis. In \figref{CE} we demonstrate 3 counter-examples from the case study. We only show the activations of the atrial node and ventricle pacing. 
$CE_{at}$ corresponds to fast intrinsic heart rate (i.e. when running). The pacemaker is pacing the ventricle after $TAVI$ period to maintain optimal A-V delay, which is a safe execution. 

$CE_{pvc}$ has a very similar execution with $CE_{at}$. However, the activations of the atrial node is triggered by retrograde conduction from ventricular paces (marker \textsf{cond}). The atrial activations trigger another ventricular pace after $TAVI$, which will trigger another retrograde conduction. In this case the heart rate is controlled by $TAVI+Tcond$, which corresponds to an inappropriate closed-loop behavior called Endless Loop Tachycardia.

In $CE_{af}$ the atrial rate is very high, which is a sub-optimal heart condition. However, the ventricular rate can stay normal due to the blocking property of the AV node. Despite the filters in the pacemaker, the pacemaker still paces the ventricle for every 3 atrial activations, which extends fast atrial rate to more dangerous fast ventricular rate. This scenario is referred to as Atrial Tachycardia Response of a pacemaker. 

From the analysis, pacemaker operations in $CE_{pvc}$ and $CE_{af}$ need to be revised. However, the revision should not affect the behavior in $CE_{at}$. This example demonstrates that counter-examples from more refined models provide more detailed mechanism of the requirement violations, and distinguish the physiological conditions that can trigger the violations. The information is helpful for debugging and improving the algorithm. The physicians can also improve the physiological requirement so that these heart conditions can be then considered case by case. %Both $CE_b$ and $CE_c$ are inappropriate executions of the pacemaker .$CE_a$ and $CE_b$ can have the same input-output executions on the pacemaker side and can only be differentiated on the heart model side. After the physician examines the counter-example the programmer can work on debugging. 
%$NA\_self$ is in $H_3$, we go one level up, in $H_4$ the behavior is not merged with any other parameters. In $H_5$ $NA\_self$ is merged with  $NA'-NV'.cond$ so $H_4$ is returned as the appropriate heart model for R1. In \cite{STTT13} we used $H_4$ to verify the correctness of the ELT termination algorithm. With a basic DDD pacemaker we have $H_4 || P_{DDD}\models R1$. The counter-example returned is exactly the ELT behavior. Then we implement the ELT termination algorithm and we have  $H_4 || P_{ELT}\not\models R1$, meaning ELT has been successfully terminated, and only the ELT is terminated. 
%
%\subsection{Inappropriate Model Refinements}
%If we follow the traditional CEGAR framework and verify the property using $H_5$, an abstract counter-example would return, which is shown in %\figref{C_amiguity}. However the counter-example correspond
 




