\section{Resolving Ambiguities Using the Abstraction Tree}
\subsection{Resolving Context Ambiguities}
\begin{Verbatim}
Algorithm 1
function [HM]=eligible(HM_tree,Req)
BM = root of HM_tree
while (BM is not empty)
 For every model M in BM
  If (all behaviors in Prop(Req) are not abstracted with other behaviors)
   Remove M from BM
   save M in HM
  else
   add children of M in BM
  endif
 endfor
endwhile
Return HM
\end{Verbatim}
%BM = all models in HM_tree with all Prop(Req)
%For all M in BM
 %while (in the parent of M no behavior in Prop(Req) is abstracted with other behaviors)
	 %M = parent of M
 %endwhile
 %save M in HM
 %endfor
\subsection{Resolving Validity Ambiguities}

\begin{Verbatim}
Algorithm 2
Input: system model PM, abstraction tree for environment HM_tree, requirement Req
Output: Counter examples CE and corresponding model refinements
[HM]=eligible(HM_tree,Req);
Mc= HM;
 while (Mc is not empty)
  For all M in Mc
   [satisfied,CE]=ModelChecking(M,PM,Req);
	 Remove M from Mc
	 If satisfied==0
	  add the children of M to Mc
		cache CE
	else
		save CE from the parent model
	endif
 endfor
endwhile
Return all saved CEs and their corresponding models
\end{Verbatim}



\section{Case Study: Closed-loop Model Checking of a Dual Chamber Pacemaker}
\label{contextAmbiguities}
\subsection{Requirement Encoding}
\begin{itemize}
	\item Pre-condition: Normal atrial self-activation rate (60bpm - 100bpm)
    \item Post-condition: Ventricular pace rate no faster than LRI
\end{itemize}
\Hao{Need a monitor here}
The requirement can be translated to:
$$R1: NA\_self.min=600,NA\_self.max=1000\Rightarrow VP-VP>=TLRI$$
$Prop(R1)=NA\_self$
\subsection{Solving Requirement Ambiguities}
Run the function $[HM]=eligible(HM\_tree,R1)$. Since 
%$NA\_self$ is in $H_3$, we go one level up, in $H_4$ the behavior is not merged with any other parameters. In $H_5$ $NA\_self$ is merged with  $NA'-NV'.cond$ so $H_4$ is returned as the appropriate heart model for R1. In \cite{STTT13} we used $H_4$ to verify the correctness of the ELT termination algorithm. With a basic DDD pacemaker we have $H_4 || P_{DDD}\models R1$. The counter-example returned is exactly the ELT behavior. Then we implement the ELT termination algorithm and we have  $H_4 || P_{ELT}\not\models R1$, meaning ELT has been successfully terminated, and only the ELT is terminated. 
%
%\subsection{Inappropriate Model Refinements}
%If we follow the traditional CEGAR framework and verify the property using $H_5$, an abstract counter-example would return, which is shown in %\figref{C_amiguity}. However the counter-example correspond
 




