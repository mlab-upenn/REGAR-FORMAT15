\subsubsection{Rule R5: Merge Self-activation Nodes with Interaction Nodes}
%\todo[inline]{not clear}
The effect of self-activation nodes on the interaction of the pacemaker is triggering sensing events within certain delay. In this rule we merge all the self-activation nodes to their nearest interaction nodes. \\
\textbf{Applicability conditions: }The rule can be applied after all the non-activating nodes are removed.\\
\textbf{Output graph: }Assume the set of sensing nodes are $N_s^i$ and the set of self-activating nodes are $N_a^j$. The minimum timing delay between two nodes are the sum of conduction delay of the path automata along the path, which we denote as $dist(N_1,N_2)$. Each self-activating nodes $N_a$ is merged with the nearest sensing node $i=arcmin_i(dist(N_s^i,N_a))$.\\
\textbf{Effects on parameters: }Assume $N_a^i$ is the set of self-activating nodes that are merged with sensing node $N_s^i$. The parameters of $N_s^i$ are decided by:
$$N_s^i.TERP\_min=min(N_a^i.TERP\_min), N_s^i.TERP\_max=max(N_a^i.TERP\_max)$$
$$N_s^i.Trest\_min=min(N_a^i.Trest\_min), N_s^i.Trest\_max=max(N_a^i.Trest\_max)$$

\subsubsection{Rule R6: Replace Blocking With Non-deterministic Conduction}
%\todo[inline]{in modeling section mention self-activating and passive nodes}
%If a node automaton is in its \textsf{ERP} state, a \textsf{Act\_node} event is "blocked" and will not trigger corresponding path conduction. If we add non-deterministic transitions to the path automata such that a \textsf{Act\_path} event do not trigger state transitions to \textsf{ante} or \textsf{retro} (\figref{rule5}), the blocking behavior of the node is covered. We can then merge the \textsf{ERP} state with the \text{Rest} state in the node automata, and all passive nodes can be removed. An application of Rule 6 is shown in \figref{abs_exam}.\\
Consider the structure $N_1 P_1 N_2 P_2 N_3$ with three nodes and two paths, where $N_2$ is a passive node (i.e. not self-activating).
If $N_2$ blocks an activation signal from $N_1$ to $N_3$, this is equivalent to the paths $P_1$ or $P_2$ not conducting.
In this rule, the structure $P_1 N_2 P_2$ is replaced by a path $P$ whose automaton can take a self loop when it receives an activation signal, thus effectively stopping the conduction. 
This is shown in Fig.~\ref{fig:rule5}: the extra transitions are marked Act\_path\_1? and Act\_path\_2?.
Because the blocking effect of nodes is now incorporated into the paths, the node automata of self-activating nodes can be modified to the one shown in Fig.~\ref{fig:rule5}, which doesn't have the (now useless) ERP period.
\\
\textbf{Subgraph to which it applies}.
Line graphs with 3 vertices $N_1 P_1 N_2 P_2 N_3$, and self-activating nodes.\\
\textbf{Applicability conditions}
$N_2$ is a passive node.\\
\textbf{Output subgraph}.
$N'_1 P' N'_3$
A path $P'$ whose path automaton is as shown in Fig.~\ref{fig:rule5}.b.
The self-activating nodes $N$ are replaced by nodes $N'$ with automata shown in Fig.~\ref{fig:rule5}.a.\\
\textbf{Effect on parameters}
For the new path, $P.cond_{min}=P_1.cond_{min}+P_2.cond_{min}$ and 
$P.cond_{max}=P_1.cond_{max}+P_2.cond_{max}$
For the new nodes, $N'.Trest_{min}=N.Terp_{min}+N.Trest_{min}$ and 
$N'.Trest_{max}=N.Terp_{max}+N.Trest_{max}$.\\
\subsubsection{Interaction With the Pacemaker}
The interactions between the heart and the pacemaker are modeled by using binary event channels. For the atrial lead, we have:
\textsf{$N_A.Act\_path!\rightarrow$AS!},
and for ventricular lead we have
\textsf{$N_V.Act\_path!\rightarrow$VS!}.\\
The pacemaker accordingly generates atrial or ventricular pacing actions \textsf{AP!$\rightarrow N_A.Act\_node!$} and \textsf{VP!$\rightarrow N_V.Act\_node!$}.