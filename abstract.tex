\begin{abstract}
This paper proposes a methodology for closed-loop model checking of medical devices, and illustrates it with a case study on implantable pacemakers.
To evaluate the performance of a medical device on the human body, a model of the device's physiological environment must be developed, and the closed loop consisting of device (e.g., pacemaker) and environment (e.g., the human heart) is model-checked.
Formal modeling of the environment and its application in model checking pose several challenges that are addressed in this paper.
Pacemakers should guarantee safe operations within large varieties of heart conditions, which are modeled by a set of timed automata models.
A set of domain-specific abstraction rules are developed that can over-approximate the timing behaviors of a heart model or a group of heart models, such that the new behaviors introduced by abstraction are mostly physiologically meaningful.
The rules serve as a systematic method to cover heart conditions that may not be explicitly accounted for in the initial set of heart models.
Automated applications of the abstraction rules result in an abstraction tree.
Closed-loop model checking is systematically performed using the heart models in the abstraction tree, to obtain the most concrete counter-example(s) correspond to property violation.
These counter-example, along with their physiological context, are then presented to the physician to determine its physiological validity.
The proposed methodology creates a separation between steps requiring physiological domain expertise (model creation and abstraction rules definition) and steps that can be automated (rule application, model checking, and abstraction refinement).
While we illustrate the methodology for pacemaker verification, it is more broadly applicable to the verification of other medical devices.

%Over-approximation can be used to not only reduce model complexity, but also cover physiological-relevant behaviors from multiple physiological conditions in an abstract model. 
%However, over-approximation inevitably introduces behaviors that are not physiologically possible, potentially causing false-negatives during model checking. Moreover, by abstracting multiple physiological conditions, behaviors from different physiological conditions become indistinguishable in over-approximated models, causing ambiguities to physicians who are responsible to determine the validity of potential counter-examples.  
%In this paper we propose an abstraction-tree-based framework for closed-loop model checking of medical devices, and use implantable pacemaker as an example.
%A set of physiological abstraction rules are developed to merge behaviors from models representing different physiological conditions, and ensure majority of behaviors added to the abstract models are also physiological-relevant.  
%By applying the physiological abstraction rules, an abstraction tree is constructed and can be used for closed-loop model checking. An automated search algorithm first select the most abstract models that are appropriate for the physiological requirements as the initial physiological models. In case of property violations, the search algorithm explore the abstraction tree for the most refined models which can still produce property violations, which branches an abstract property violation into possible physiological conditions thus provides helpful physiological context to the physicians.
%With this framework, a person with expertise in formal method do not need physiological knowledge to use the abstraction tree for model checking, and model checking results can be feedback to the physicians with unambiguous physiological context. The framework can also be extended to other Cyber Physical System domains to bridge the gap between the cyber domain and the physical domain.

\end{abstract}