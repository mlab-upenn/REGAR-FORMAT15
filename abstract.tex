\begin{abstract}
This paper proposes a methodology for formal verification of medical devices, and illustrates it with a case study in verifying pacemakers.
To understand the effect of an implantable medical device on the human body, a model of the device's physiological environment must be developed, and the closed loop consisting of device (e.g., pacemaker) and environment (e.g., the human heart) is model-checked.
Formal modeling and model checking of the environment poses several challenges that are addressed in this paper.
First, we develop a set of timed automata models of various heart conditions and anomalies.
We then introduce a set of domain-specific abstraction rules that abstract from these models, such that the new behavior introduced by abstraction is physiologically meaningful, and not necessarily spurious as in traditional abstraction.
Thus the rules serve as a systematic method to model new conditions not explicitly accounted for in the initial set of models.
Because we abstract from several initial models, we obtain an abstraction tree.
This tree is then used to perform a model checking loop where we seek the most concrete model that displays a given counter-example.
This counter-example is then presented to the physician along with the heart model that generated it, so the physician can determine whether it is physiologically plausible or not.
The proposed methodology creates a separation between steps requiring domain expertise (model creation and abstraction rules definition) and steps that can be automated (rule application, model checking, and abstraction refinement).
While we illustrate the methodology for pacemaker verification, it is more broadly applicable to the verification of other medical devices.

%Over-approximation can be used to not only reduce model complexity, but also cover physiological-relevant behaviors from multiple physiological conditions in an abstract model. 
%However, over-approximation inevitably introduces behaviors that are not physiologically possible, potentially causing false-negatives during model checking. Moreover, by abstracting multiple physiological conditions, behaviors from different physiological conditions become indistinguishable in over-approximated models, causing ambiguities to physicians who are responsible to determine the validity of potential counter-examples.  
%In this paper we propose an abstraction-tree-based framework for closed-loop model checking of medical devices, and use implantable pacemaker as an example.
%A set of physiological abstraction rules are developed to merge behaviors from models representing different physiological conditions, and ensure majority of behaviors added to the abstract models are also physiological-relevant.  
%By applying the physiological abstraction rules, an abstraction tree is constructed and can be used for closed-loop model checking. An automated search algorithm first select the most abstract models that are appropriate for the physiological requirements as the initial physiological models. In case of property violations, the search algorithm explore the abstraction tree for the most refined models which can still produce property violations, which branches an abstract property violation into possible physiological conditions thus provides helpful physiological context to the physicians.
%With this framework, a person with expertise in formal method do not need physiological knowledge to use the abstraction tree for model checking, and model checking results can be feedback to the physicians with unambiguous physiological context. The framework can also be extended to other Cyber Physical System domains to bridge the gap between the cyber domain and the physical domain.

\end{abstract}