\begin{abstract}
Implantable medical devices are designed to diagnose and improve certain adverse physiological conditions, thus the safety and efficacy of the devices have to be evaluated in closed-loop within their physiological context. 
Model-based design has enabled closed-loop verification early in the design stage and closed-loop model checking has been proposed to provide confidence in the safety and efficacy of the devices. 
The biggest challenge for closed-loop model-checking of medical devices is modeling the physiological environment. Unlike system modeling in which there is only one concrete system, there are countless number of physiological conditions and different models are required to distinguish behaviors associated with each condition.
Over-approximation can be used to not only reduce model complexity, but also cover physiological-relevant behaviors from multiple physiological conditions in an abstract model. %If the closed-loop model with the device model and the abstract environment model satisfy a requirement, the device model satisfies the physiological requirement under all physiological conditions covered by the abstract environment model.
However, over-approximation inevitably introduce behaviors that are not valid in all physiological conditions, potentially causing false-negatives during model checking. Moreover, by abstracting multiple physiological conditions, behaviors from different physiological conditions become indistinguishable in an over-approximated model, causing ambiguities.  %there are two challenges for using over-approximated environment model in closed-loop model checking. The first one lose its physiological-relevance,introduced
In this paper we propose an abstraction tree
%During closed-loop model checking, both the system and its environment model are abstracted The Counter-Example-Guided Abstraction and Refinement (CEGAR) framework has been developed to reduce the complexity of the model using over-approximation and refine the model when false-positives are found. However in environment modeling, an  therefore determining the validity of a closed-loop execution is not intuitive, making it difficult to apply CEGAR during environment modeling. On the other hand, the development of environment models and validation of closed-loop executions rely heavily on domain expertise. It is therefore possible to encode the domain knowledge during environment modeling and provide guidance during the refinement of the environment models. In this paper, we use implantable pacemaker as example to illustrate the challenges for applying CEGAR during environment modeling, and how to use encoded domain knowledge to guide the environment model abstraction and refinement.
\end{abstract}