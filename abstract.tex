\begin{abstract}
Implantable medical devices are designed to diagnose and improve certain adverse physiological conditions, thus the safety and efficacy of the devices have to be evaluated in closed-loop within their physiological context. Model-based design has enabled closed-loop verification early in the design stage and closed-loop model checking has been proposed to provide confidence in the safety and efficacy of the devices. One of the biggest challenge for model-checking is to balance the complexity and the expressiveness of the models. During closed-loop model checking, both the system and its environment model are abstracted The Counter-Example-Guided Abstraction and Refinement (CEGAR) framework has been developed to reduce the complexity of the model using over-approximation and refine the model when false-positives are found. However in environment modeling, an environment model generally abstracts a large number of environment conditions, therefore determining the validity of a closed-loop execution is not intuitive, making it difficult to apply CEGAR during environment modeling. On the other hand, the development of environment models and validation of closed-loop executions rely heavily on domain expertise. It is therefore possible to encode the domain knowledge during environment modeling and provide guidance during the refinement of the environment models. In this paper, we use implantable pacemaker as example to illustrate the challenges for applying CEGAR during environment modeling, and how to use encoded domain knowledge to guide the environment model abstraction and refinement.
\end{abstract}