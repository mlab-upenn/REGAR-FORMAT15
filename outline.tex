\section{outline}
\begin{enumerate}
	\item Introduction
	\subitem The problem of closed-loop formal verification
	\item In this paper, we illustrate four challenges and solve them for medical devices, specifically cardiac devices.
	\subitem Some elements of the proposed method can be generalized and have a wider applicability: namely the abstraction tree and MC on the tree (abstractness measure, search procedure)
	\subitem Give overview of method in a block diagram and briefly explain it in text.
	
	\item Formal models of the environment
	\begin{enumerate}
		\item Heart basics, enough to understand the proposed models
		\item Emphasis: different heart conditions necessitate different models.
		\item Creation of formal heart models: graph structure + node and path automata
		\item Two examples of heart conditions and their formal models.
		\item Emphasis: our initial set of models is not necessarily complete. That is, there are more heart conditions that could be modeled, but aren't. It is always a work in progress.		
		\item Formalization of physiological requirements
	\end{enumerate}
	
	\item Abstraction rules
	\begin{enumerate}
		\item Why predicate-based abstraction can be inadequate? Example
		\label{ar:inadequate}
		\item Physiologically meaningful abstraction rules
		\subitem Give formal definition as a graph-to-graph function $R: G \rightarrow G'$ 
		\subitem Why do you call it abstraction? 
		\subitem Given 2 concrete examples, with 2 more in appendix (more in tech report?)
		\item Example of rule application to example \ref{ar:inadequate}.
		\item Abstraction tree		
	\end{enumerate}
	
	\item Model-checking with the abstraction tree
	\begin{enumerate}
		\item Measure of abstractness; define 'appropriate for the requirement'
		\item Search procedure
		\item What happens if a counter-example is found
		\item Elaborate example
	\end{enumerate}
	
	\item Conclusions and future research
	
\end{enumerate}


