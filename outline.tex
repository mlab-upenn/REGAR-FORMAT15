\section{outline}

\begin{enumerate}
	\item Introduction
	\subitem The difference between software specification and physiological requirement
	\subitem The necessity of closed-loop verification
	\subitem Model-based design enables closed-loop verification at earlier design stage.
	\subitem Closed-loop model checking
	\subitem Difference between system modeling and environment modeling
	\subsubitem One concrete system vs. countless concrete systems
	\subsubitem Abstraction for simplicity vs. generality
	%\subitem In this paper, we illustrate four challenges and solve them for medical devices, specifically cardiac devices. 
	\subitem CEGAR does not work on environment modeling
	\subsubitem Physiological requirements focus on environment conditions, thus predicate abstraction does not work (example)
	\subsubitem Random abstraction functions introduce behaviors that are not physiological possible
	\subsubitem Validity of counter-examples cannot be checked
	\subitem Abstraction tree based environment model abstraction and refinement
	\subsubitem Physiological abstraction rules introduce physiological behaviors into the abstract model
	\subsubitem Documenting merging of physiological transitions to maintain the physiological-relevance of the abstract model
	\subsubitem Abstraction tree keeps track of abstraction-refinement relations
	\subsubitem Simple algorithm to select appropriate initial model for requirements
	\subsubitem Algorithm to obtain the most concrete counter-examples corresponding to multiple physiological conditions.
	\subsubitem Physician determine the validity of the counter-examples
	\subitem Some elements of the proposed method can be generalized and have a wider applicability: namely the abstraction tree and MC on the tree (abstractness measure, search procedure)
	\subitem Give overview of method in a block diagram and briefly explain it in text.
	
	\item Formal models of the environment
	\begin{enumerate}
		\item Heart basics, enough to understand the proposed models
		\item Emphasis: different heart conditions necessitate different models.
		\item Creation of initial formal heart models: graph structure + node and path automata
		%\item Two examples of heart conditions and their formal models.
		\item Emphasis: our initial set of models is not necessarily complete. That is, there are more heart conditions that could be modeled, but aren't. It is always a work in progress.	
		\subitem Therefore we want to add behavior to try and cover things not covered by the models. Adding models i not the answer since it will always be an incomplete set. Not to mention it's a manual process.
		\item Formalization of physiological requirements
	\end{enumerate}
	
	\item Abstraction rules: how to add behavior to the initial set of models
	\begin{enumerate}
		\item Why predicate-based abstraction can be inadequate?
		%\label{ar:inadequate}
		\subitem If we get a cex which is spurious by the classical definition, it is not necessarily spurious physiologically. Remember we are enriching the behavior precisely to cover new conditions.
		\subitem So it's up to the physician to decide if the cex is spurious or not. But predicate-based abstraction might give hard-to-understand cex.
%		\subsubitem Two ambiguities: interaction and context ambiguities. Interaction ambiguity because the requirement is purely on the env.
		\subitem Example
		\subitem Therefore, need the added behavior to be physiologically meaningful so it's understandable.
		\item Physiologically meaningful abstraction rules
		\subitem Give formal definition of a rule as a graph-to-graph function $R: G \rightarrow G'$ 
		\subitem Why do you call it abstraction? 
		\subitem Given 2 concrete examples (more in tech report)
		\item Example of rule application.
		\item Abstraction tree		
	\end{enumerate}
	
	\item Model-checking with the abstraction tree
	\begin{enumerate}
		\item Measure of abstractness; define 'appropriate for the requirement'
		\item Search procedure
		\item What happens if a counter-example is found: give the physician the most concrete model on which the cex still shows up.
		\item Case study: i.e., elaborate example
	\end{enumerate}
	
	\item Conclusions and future research
	
\end{enumerate}


